\documentclass{article}

\usepackage[english]{babel}
\usepackage[utf8]{inputenc}
\usepackage{amsmath,amssymb}
\usepackage{parskip}
\usepackage{graphicx}
\usepackage{listings}
\usepackage{float}

% Margins
\usepackage[top=2.5cm, left=3cm, right=3cm, bottom=4.0cm]{geometry}
% Colour table cells
\usepackage[table]{xcolor}

% Get larger line spacing in table
\newcommand{\tablespace}{\\[1.25mm]}
\newcommand\Tstrut{\rule{0pt}{2.6ex}}         % = `top' strut
\newcommand\tstrut{\rule{0pt}{2.0ex}}         % = `top' strut
\newcommand\Bstrut{\rule[-0.9ex]{0pt}{0pt}}   % = `bottom' strut

%%%%%%%%%%%%%%%%%
%     Title     %
%%%%%%%%%%%%%%%%%
\title{CSCI946 Assignment}
\author{Yao Xiao \\ SID 2019180015}
\date{\today}

\begin{document}
\maketitle

%%%%%%%%%%%%%%%%%
%   Problem 1   %
%%%%%%%%%%%%%%%%%
\section{Phase 1: Discovery}
\begin{enumerate}
    \item Identify data source.
    \item Confirm team members and roles.
    \item Check project sponsor's approach.
\end{enumerate}

\subsection{About Data}
The data for the project should be classified into two categories:
\begin{enumerate}
    \item Five years of ideas submitted by internal innovation competitions.
    \item Notes and records representing innovation and research activities from all over the world.
\end{enumerate}

\subsection{Hypotheses}
The hypotheses for the project should be grouped into two categories:
\begin{enumerate}
    \item Descriptive analysis of what is happening to further stimulate related creativity and collaboration.
    \item Predictive analysis can provide relevant administrative staff with future investment recommendations, such as how much to invest at what time, etc.
\end{enumerate}

\section{Phase 2: Data Preparation}
\begin{enumerate}
    \item IT department set up an analytics sandbox
    \item Discovered that certain data needed conditioning and normalization and that missing datasets were critical.
    \item Many names were misspelled and problems with extra spaces.
\end{enumerate}

\section{Phase 3: Model Planning}
\begin{enumerate}
    \item Use social network analysis techniques to look for innovators.
    \item Identify the right milestones to achieve the goals.
    \item Trace how people move ideas from each milesonte toward the goal.
    \item Tract ideas that die and others that reach the goal.
    \item Compare times and outcomes using a few different methods.
\end{enumerate}

\section{Phase 4: Model Building}
\begin{enumerate}
    \item NLP on textual descriptions.
    \item Conduct social network analysis using R and Rstudio.
    \item Developed social graphs and visualizations.
    \item Verify some of the inital hypotheses.
\end{enumerate}

\section{Phase 5: Communicate Results}
\begin{enumerate}
    \item Study was successful in in identifying hidden innovators.
    \item CTO launched longitudinal studies to track innovation results.
    \item This project promoted knowledge sharing related to innovation and research.
    \item Also enable the organisation to cultivate additional IP.
    \item Communicate one of the key findings.
\end{enumerate}

\section{Phase 6: Operationalize}
\begin{enumerate}
    \item Deployment was not really discussed.
    \item Key findings:
    \begin{itemize}
        \item Need more data in future.
        \item Some data were sensitive.
        \item A parallel initiative needs to be created to improve basic BI activities.
        \item A mechanism is needed to continually reevaluate the model after deployment.
        \item Informed new investment decisions.s
    \end{itemize}
\end{enumerate}


\end{document}
