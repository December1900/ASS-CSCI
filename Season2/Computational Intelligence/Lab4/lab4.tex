%=======================02-713 LaTeX template, following the 15-210 template==================
%
% You don't need to use LaTeX or this template, but you must turn your homework in as
% a typeset PDF somehow.
%
% How to use:
%    1. Update your information in section "A" below
%    2. Write your answers in section "B" below. Precede answers for all 
%       parts of a question with the command "\question{n}{desc}" where n is
%       the question number and "desc" is a short, one-line description of 
%       the problem. There is no need to restate the problem.
%    3. If a question has multiple parts, precede the answer to part x with the
%       command "\part{x}".
%    4. If a problem asks you to design an algorithm, use the commands
%       \algorithm, \correctness, \runtime to precede your discussion of the 
%       description of the algorithm, its correctness, and its running time, respectively.
%    5. You can include graphics by using the command \includegraphics{FILENAME}
%
\documentclass[11pt]{article}
\usepackage{amsmath,amssymb,amsthm}
\usepackage{graphicx}
\usepackage[margin=1in]{geometry}
\usepackage{fancyhdr}
\usepackage{listings}
\setlength{\parindent}{0pt}
\setlength{\parskip}{5pt plus 1pt}
\setlength{\headheight}{13.6pt}
\newcommand\question[2]{\vspace{.25in}\hrule\textbf{#1: #2}\vspace{.5em}\hrule\vspace{.10in}}
\renewcommand\part[1]{\vspace{.10in}\textbf{(#1)}}
\newcommand\algorithm{\vspace{.10in}\textbf{Algorithm: }}
\newcommand\correctness{\vspace{.10in}\textbf{Output: }}
\newcommand\runtime{\vspace{.10in}\textbf{Running time: }}
\pagestyle{fancyplain}
\lhead{\textbf{\NAME\ (\ANDREWID)}}
\chead{\textbf{Lab\HWNUM}}
\rhead{\today}
\begin{document}\raggedright
%Section A==============Change the values below to match your information==================
\newcommand\NAME{Yao Xiao}  % your name
\newcommand\ANDREWID{2019180015}     % your andrew id
\newcommand\HWNUM{4}              % the homework number
%Section B==============Put your answers to the questions below here=======================

% no need to restate the problem --- the graders know which problem is which,
% but replacing "The First Problem" with a short phrase will help you remember
% which problem this is when you read over your homeworks to study.

\question{1}{The First Problem} 

\part{a} \algorithm
\begin{lstlisting}
from sklearn import datasets
from sklearn import svm
import numpy as np

iris = datasets.load_iris()
print(type(iris), dir(iris))

x = iris.data
y = iris.target

data = x.shape[0]
ratio = 8/2
data_test = int(data / (1 + ratio))
data_train = data - data_test
index = np.arange(data)
np.random.shuffle(index)

x_test = x[index[:data_test],:] 
y_test = y[index[:data_test]]
x_train = x[index[data_test:],:] 
y_train = y[index[data_test:]]

svm_linear = svm.SVC(decision_function_shape="ovo", kernel="linear")
svm_poly = svm.SVC(decision_function_shape="ovo", kernel="poly")
svm_rbf = svm.SVC(decision_function_shape="ovo", kernel="rbf")

svm_linear.fit(x_train, y_train)
svm_poly.fit(x_train, y_train)
svm_rbf.fit(x_train, y_train)
pre_linear_y_test = svm_linear.predict(x_test)
pre_poly_y_test = svm_poly.predict(x_test)
pre_rbf_y_test = svm_rbf.predict(x_test)


acc_linear = sum(pre_linear_y_test == y_test) / data_test
print('The accuracy of linear kernel is: ', acc_linear)
acc_poly = sum(pre_poly_y_test == y_test) / data_test
print('The accuracy of poly kernel is: ', acc_poly)
acc_rbf = sum(pre_rbf_y_test == y_test) / data_test
print('The accuracy of rbf kernel is: ', acc_rbf)
\end{lstlisting}

\part{b} \correctness\\
\includegraphics[scale=0.8]{ot.png}



\end{document}
