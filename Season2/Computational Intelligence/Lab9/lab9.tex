%=======================02-713 LaTeX template, following the 15-210 template==================
%
% You don't need to use LaTeX or this template, but you must turn your homework in as
% a typeset PDF somehow.
%
% How to use:
%    1. Update your information in section "A" below
%    2. Write your answers in section "B" below. Precede answers for all 
%       parts of a question with the command "\question{n}{desc}" where n is
%       the question number and "desc" is a short, one-line description of 
%       the problem. There is no need to restate the problem.
%    3. If a question has multiple parts, precede the answer to part x with the
%       command "\part{x}".
%    4. If a problem asks you to design an algorithm, use the commands
%       \algorithm, \correctness, \runtime to precede your discussion of the 
%       description of the algorithm, its correctness, and its running time, respectively.
%    5. You can include graphics by using the command \includegraphics{FILENAME}
%
\documentclass[11pt]{article}
\usepackage{amsmath,amssymb,amsthm}
\usepackage{graphicx}
\usepackage[margin=1in]{geometry}
\usepackage{fancyhdr}
\usepackage{listings}
\usepackage{float} 
\usepackage{subfig}
\usepackage{pdfpages}

\setlength{\parindent}{0pt}
\setlength{\parskip}{5pt plus 1pt}
\setlength{\headheight}{13.6pt}
\newcommand\question[2]{\vspace{.25in}\hrule\textbf{#1: #2}\vspace{.5em}\hrule\vspace{.10in}}
\renewcommand\part[1]{\vspace{.10in}\textbf{(#1)}}
\newcommand\algorithm{\vspace{.10in}\textbf{Algorithm: }}
\newcommand\ot{\vspace{.10in}\textbf{Output: }}
\newcommand\runtime{\vspace{.10in}\textbf{Running time: }}
\pagestyle{fancyplain}
\lhead{\textbf{\NAME\ (\ANDREWID)}}
\chead{\textbf{Lab\HWNUM}}
\rhead{\today}
\begin{document}\raggedright
%Section A==============Change the values below to match your information==================
\newcommand\NAME{Yao Xiao}  % your name
\newcommand\ANDREWID{2019180015}     % your andrew id
\newcommand\HWNUM{9}              % the homework number
%Section B==============Put your answers to the questions below here=======================

% no need to restate the problem --- the graders know which problem is which,
% but replacing "The First Problem" with a short phrase will help you remember
% which problem this is when you read over your homeworks to study.

\question{1}{The First Problem} 

\part{a} \algorithm
\begin{lstlisting}
import numpy as np
import os
import sys
import tensorflow as tf
from tensorflow import keras
from tensorflow.keras.layers import Conv2D, MaxPooling2D, Dropout, Flatten, Dense
from tensorflow.keras import backend as K
from tensorflow.keras.models import Sequential
from tensorflow.keras.datasets import mnist
import matplotlib.pyplot as plt
import pandas as pd
import warnings
warnings.filterwarnings("ignore")

batch_size = 128
num_classes = 10
epochs = 10

img_rows, img_cols = 28, 28

(x_train, y_train), (x_test, y_test) = mnist.load_data()

if K.image_data_format() == 'channels_first':
    x_train = x_train.reshape(x_train.shape[0], 1, img_rows, img_cols)
    x_test = x_test.reshape(x_test.shape[0], 1, img_rows, img_cols)
    input_shape = (1, img_rows, img_cols)
else:
    x_train = x_train.reshape(x_train.shape[0], img_rows, img_cols, 1)
    x_test = x_test.reshape(x_test.shape[0], img_rows, img_cols, 1)
    input_shape = (img_rows, img_cols, 1)

x_train = x_train.astype('float32')
x_test = x_test.astype('float32')
x_train /= 255
x_test /= 255

print('train samples: ', x_train.shape[0])
print( 'test samples: ', x_test.shape[0])

y_train = keras.utils.to_categorical(y_train, num_classes)
y_test = keras.utils.to_categorical(y_test, num_classes)

model = Sequential()
model.add(Conv2D(32, kernel_size=(3, 3),
                 activation='relu',
                 input_shape=input_shape))
model.add(Conv2D(64, (3, 3), activation='relu'))
model.add(MaxPooling2D(pool_size=(2, 2)))
model.add(Dropout(0.25))
model.add(Flatten())
model.add(Dense(128, activation='relu'))
model.add(Dropout(0.5))
model.add(Dense(num_classes, activation='softmax'))

model.compile(loss=keras.losses.categorical_crossentropy,
              optimizer=keras.optimizers.Adadelta(),
              metrics=['accuracy'])

model.fit(x_train, y_train,
          batch_size=batch_size,
          epochs=epochs,
          verbose=1,
          validation_data=(x_test, y_test))

          score = model.evaluate(x_test, y_test, verbose=0)
          print('Test loss:', score[0])
          print('Test acc:', score[1])


\end{lstlisting}

\part{b} \ot

\includegraphics[scale=0.4]{Fig1}


\end{document}
