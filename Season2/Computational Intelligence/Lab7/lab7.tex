%=======================02-713 LaTeX template, following the 15-210 template==================
%
% You don't need to use LaTeX or this template, but you must turn your homework in as
% a typeset PDF somehow.
%
% How to use:
%    1. Update your information in section "A" below
%    2. Write your answers in section "B" below. Precede answers for all 
%       parts of a question with the command "\question{n}{desc}" where n is
%       the question number and "desc" is a short, one-line description of 
%       the problem. There is no need to restate the problem.
%    3. If a question has multiple parts, precede the answer to part x with the
%       command "\part{x}".
%    4. If a problem asks you to design an algorithm, use the commands
%       \algorithm, \correctness, \runtime to precede your discussion of the 
%       description of the algorithm, its correctness, and its running time, respectively.
%    5. You can include graphics by using the command \includegraphics{FILENAME}
%
\documentclass[11pt]{article}
\usepackage{amsmath,amssymb,amsthm}
\usepackage{graphicx}
\usepackage[margin=1in]{geometry}
\usepackage{fancyhdr}
\usepackage{listings}
\usepackage{float} 
\usepackage{subfig}
\usepackage{pdfpages}

\setlength{\parindent}{0pt}
\setlength{\parskip}{5pt plus 1pt}
\setlength{\headheight}{13.6pt}
\newcommand\question[2]{\vspace{.25in}\hrule\textbf{#1: #2}\vspace{.5em}\hrule\vspace{.10in}}
\renewcommand\part[1]{\vspace{.10in}\textbf{(#1)}}
\newcommand\algorithm{\vspace{.10in}\textbf{Algorithm: }}
\newcommand\ot{\vspace{.10in}\textbf{Output: }}
\newcommand\runtime{\vspace{.10in}\textbf{Running time: }}
\pagestyle{fancyplain}
\lhead{\textbf{\NAME\ (\ANDREWID)}}
\chead{\textbf{Lab\HWNUM}}
\rhead{\today}
\begin{document}\raggedright
%Section A==============Change the values below to match your information==================
\newcommand\NAME{Yao Xiao}  % your name
\newcommand\ANDREWID{2019180015}     % your andrew id
\newcommand\HWNUM{7}              % the homework number
%Section B==============Put your answers to the questions below here=======================

% no need to restate the problem --- the graders know which problem is which,
% but replacing "The First Problem" with a short phrase will help you remember
% which problem this is when you read over your homeworks to study.

\question{1}{The First Problem} 

\part{a} \algorithm
\begin{lstlisting}
import numpy as np
import pandas as pd
import matplotlib.pyplot as plt
from sklearn.model_selection import train_test_split
from sklearn.linear_model import LogisticRegression
from sklearn.svm import SVC
from sklearn.metrics import accuracy_score, f1_score, precision_score, recall_score, classification_report, confusion_matrix
from sklearn.tree import DecisionTreeClassifier

car = pd.read_csv('car.csv', header=None)
car.columns = ['buying','maint','doors','persons','lug_boot','safety','evaluation']
car.describe()

x = car.iloc[:, :-1]
y = car.iloc[:, 6]
x.columns = ['buying','maint','doors','persons','lug_boot','safety']
y.columns = ['evaluation']
x.head()

x = pd.get_dummies(x, prefix_sep='_')
x = x.values
y = y.values
x_train, x_test, y_train, y_test = train_test_split(x, y, test_size=0.2, random_state=0)

# Logistic regression
clf = LogisticRegression(random_state=0)
clf.fit(x_train, y_train)
y_pred = clf.predict(x_test)
print("Training Acc: ",clf.score(x_train, y_train))
print("Testing Acc: ", clf.score(x_test, y_test))
cm = confusion_matrix(y_test, y_pred)
print(cm)
print(classification_report(y_test,y_pred))

# Linear SVC
clf = SVC(kernel='linear', random_state=0)
clf.fit(x_train, y_train)
y_pred = clf.predict(x_test)
print("Training Acc: ",clf.score(x_train, y_train))
print("Testing Acc: ", clf.score(x_test, y_test))
cm = confusion_matrix(y_test, y_pred)
print(cm)
print(classification_report(y_test,y_pred))

print('==========optimize==========\n')
clf = SVC(C=1.21, kernel='linear', random_state=0, tol=0.008)
clf.fit(x_train, y_train)
y_pred = clf.predict(x_test)
print("Training Acc: ",clf.score(x_train, y_train))
print("Testing Acc: ", clf.score(x_test, y_test))
cm = confusion_matrix(y_test, y_pred)
print(cm)
print(classification_report(y_test,y_pred))

# Rbf SVC
clf = SVC(kernel = 'rbf', random_state = 0)
clf.fit(x_train, y_train)
y_pred = clf.predict(x_test)
print("Training Acc: ",clf.score(x_train, y_train))
print("Testing Acc: ", clf.score(x_test, y_test))
cm = confusion_matrix(y_test, y_pred)
print(cm)
print(classification_report(y_test,y_pred))

print('==========optimize==========\n')

clf = SVC(C=1.21, kernel = 'rbf', random_state = 0, tol=0.99)
clf.fit(x_train, y_train)
y_pred = clf.predict(x_test)
print("Training Acc: ",clf.score(x_train, y_train))
print("Testing Acc: ", clf.score(x_test, y_test))
cm = confusion_matrix(y_test, y_pred)
print(cm)
print(classification_report(y_test,y_pred))

# DT
clf = DecisionTreeClassifier(criterion = 'entropy', random_state = 0)
clf.fit(x_train, y_train)
y_pred = clf.predict(x_test)
print("Training Acc: ",clf.score(x_train, y_train))
print("Testing Acc: ", clf.score(x_test, y_test))
cm = confusion_matrix(y_test, y_pred)
print(cm)
print(classification_report(y_test,y_pred))
\end{lstlisting}


\part{b} \ot

\includepdf[pages={1,2,3,4,5}]{lab.pdf}


\end{document}
