%=======================02-713 LaTeX template, following the 15-210 template==================
%
% You don't need to use LaTeX or this template, but you must turn your homework in as
% a typeset PDF somehow.
%
% How to use:
%    1. Update your information in section "A" below
%    2. Write your answers in section "B" below. Precede answers for all 
%       parts of a question with the command "\question{n}{desc}" where n is
%       the question number and "desc" is a short, one-line description of 
%       the problem. There is no need to restate the problem.
%    3. If a question has multiple parts, precede the answer to part x with the
%       command "\part{x}".
%    4. If a problem asks you to design an algorithm, use the commands
%       \algorithm, \correctness, \runtime to precede your discussion of the 
%       description of the algorithm, its correctness, and its running time, respectively.
%    5. You can include graphics by using the command \includegraphics{FILENAME}
%
\documentclass[11pt]{article}
\usepackage{amsmath,amssymb,amsthm}
\usepackage{graphicx}
\usepackage[margin=1in]{geometry}
\usepackage{fancyhdr}
\setlength{\parindent}{0pt}
\setlength{\parskip}{5pt plus 1pt}
\setlength{\headheight}{13.6pt}
\newcommand\question[2]{\vspace{.25in}\hrule\textbf{#1: #2}\vspace{.5em}\hrule\vspace{.10in}}
\renewcommand\part[1]{\vspace{.10in}\textbf{(#1)}}
\newcommand\algorithm{\vspace{.10in}\textbf{Algorithm: }}
\newcommand\correctness{\vspace{.10in}\textbf{Correctness: }}
\newcommand\runtime{\vspace{.10in}\textbf{Running time: }}
\pagestyle{fancyplain}
\lhead{\textbf{\NAME\ (\ANDREWID)}}
\chead{\textbf{Lab\HWNUM}}
\rhead{\today}
\begin{document}\raggedright
%Section A==============Change the values below to match your information==================
\newcommand\NAME{Yao Xiao}  % your name
\newcommand\ANDREWID{2019180015}     % your andrew id
\newcommand\HWNUM{4}              % the homework number
%Section B==============Put your answers to the questions below here=======================

% no need to restate the problem --- the graders know which problem is which,
% but replacing "The First Problem" with a short phrase will help you remember
% which problem this is when you read over your homeworks to study.

\question{1}{The First Problem}

Examine the current baseline in the Tracking Gantt view with the timescale set to weekly.\\
\includegraphics[scale=0.35]{p1-1.png}

After the plan baseline is updated, the baseline dates align with the ask dates and task 18 now has a baseline\\
\includegraphics[scale=0.35]{p1-2.png}



\question{2}{The Second Problem}

Record 92 hours of actual work on task 18, original art review.\\
\includegraphics[scale=0.45]{p2-1.png}

Change Hany Morcos’s 46 hours of actual work to 62 hours, and the other assignment to the same task is not affected.\\
\includegraphics[scale=0.43]{p2-2.png}


\question{3}{The Third Problem}

Record 9 hours of actual work on task 22 for Wednesday, May 30, and 15 hours for Thursday, May 31.\\
\includegraphics[scale=0.5]{p3-1.png}

Record 12 hours of actual work for Dan Jump’s assignment to task 22, Organize manuscript, for the week of June 3.\\
\includegraphics[scale=0.5]{p3-2.png}


\question{4}{The Fourth Problem}

Reschedule incomplete work for the entire project to start after August 7, 2018.\\

\includegraphics[scale=0.4]{p4-1.png}
\includegraphics[scale=0.35]{p4-2.png}





\end{document}
